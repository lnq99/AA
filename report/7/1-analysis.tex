\chapter{Аналитический раздел}
\label{cha:analysis}

В данном разделе представлены теоретические сведения о
алгоритмах поиска в словаре.

\section{Алгоритм полного перебора}

Алгоритм перебирает ключи словаря, пока не будет найден искомый ключ.
Возможно (N + 1) случаев: ключ не найден и N возможных случаев расположения ключа в словаре.
Лучший случай: за одно сравнение ключ найден в начале словаря.
Худших случаев два: за N сравнений либо элемент не найден, либо ключ найден на последнем сравнении.
Трудоемкость в среднем:
\begin{equation}
    \sum\limits_{i\in\Omega} p_i\cdot f_i =
    k_0+k_1(1+\frac{N}{2}-\frac{1}{N+1})
\end{equation}


\section{Алгоритм двоичного поиска}

Алгоритм требует ключи словаря отсортированы.
При двоичном поиске обход можно представить деревом, поэтому трудоемкость в худшем случае составит $log_2N$
(в худшем случае нужно спуститься по двоичному дереву от корня до листа).
Скорость роста функции $log_2N$ меньше чем у N.


\section{Алгоритм частотного анализа}

Алгоритм на вход получает словарь и на его основе составляется частотный анализ. По полученным значениям словарь разбивается на сегменты так, что все элементы с одинаковым первым элементом оказываются в одном сегменте. 

Сегменты упорядочиваются по значению частотной характеристики так, чтобы к элементы с наибольшей частотной характеристикой был самый быстрый доступ. 

Далее каждый сегмент упорядочивается по значению. Это необходимо для реализации бинарного поиска, который обеспечит эффективный поиск в сегменте при сложности $O(\log n)$.

Таким образом, сначала выбирается нужный сегмент, а затем в нем проводится бинарный поиск нужного элемента. Средняя трудоёмкость при длине алфавита $M$ может быть рассчитана по формуле \ref{eq:tfa}. 

\begin{equation} \label{eq:tfa}
	\sum\limits_{i\in[1, M]} (f_{\text{выбор i-го сегмента}}+f_{\text{поиск в i-ом сегменте}}) \cdot p_i
\end{equation}


\section{Описание словаря}

Словарь представляет собой набор информации о 1000 распространенных криптовалютах на 24.01.21.
Запись словаря, реализованная на данной работе, имеет вид
( Rank: int, Name: string, Symbol: string, Market Cap: int64, Price: float32 ).
Поиск по полю Name.

\section{Вывод}
В данном разделе были описаны два алгоритма и способ оптимизации для поиска в словаре.
Так же был рассмотрен описание словаря.