\chapter*{Введение}
\addcontentsline{toc}{chapter}{Введение}


Параллельные вычисления - способ организации компьютерных вычислений,
при котором программы разрабатываются как набор взаимодействующих вычислительных процессов, работающих параллельно (одновременно).

Конвейерная обработка улучшает использование аппаратных ресурсов для заданного набора процессов, каждый из которых применяет эти ресурсы заранее предусмотренным способом. Хорошим примером конвейерной организации является сборочный транспортер на производстве, на котором изделие последовательно проходит все стадии вплоть до готового продукта.
\\


\textbf{Целью работы:} Реализация конвейера с использованием параллельных вычислений.
\\

\textbf{Задачи работы:}

\begin{enumerate}
    \setlength{\itemsep}{0em}
    \item изучение основ конвейерной обработки данных;
    \item получение практических навыков конвейерных вычислений;
    \item экспериментальное подтверждение различий во временной
    эффективности реализаций на материале замеров процессорного времени выполнения;
\end{enumerate}
