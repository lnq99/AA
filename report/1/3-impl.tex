file:///home/ql/5/AA/report/1/4-impl.tex {"mtime":1602942489127,"ctime":1600808002510,"size":2802,"etag":"35qnnhsac2sc","orphaned":false}
\chapter{Технологический раздел}
\label{cha:impl}

\section{Требования к программному обеспечению}

Программа создана в формате записной книжки, в интерактивном режиме пользователь вводит команду в соответствии с инструкциями.
ПО должно иметь сравнение времени работы алгоритмов и должен быть хорошо протестирован.
(ПО может быть легко использован программистом)


\section{Средства реализации}

Язык программирования: Python (IPython)

Библиотеки: unittest, timeit, matplotlib, ...

Редактор: Jupyter-Lab

Я использую эти инструменты потому, что они мощные, широко используемые и знакомые мне.


\section{Листинг кода}

Примечание: я создаю таблицы и инициализирую значение в классе
Lev - interface


\lstinputlisting[
    language=python,linerange={0-9},
    caption=Рекурсивная реализация алгоритма Левенштейна
    ]{1/inc/code.py}

\hbox{}

\lstinputlisting[
    language=python,linerange={12-21},
    caption=Матричная реализация алгоритма Левенштейна
    ]{1/inc/code.py}

\hbox{}

\lstinputlisting[
    language=python,linerange={24-38},
    caption=Рекурсивная реализация алгоритма Левенштейна с заполнением матрицы
    ]{1/inc/code.py}

\hbox{}

\lstinputlisting[
    language=python,linerange={41-53},
    caption=Матричная реализация алгоритма Дамерау-Левенштейна
    ]{1/inc/code.py}

\pagebreak
\lstinputlisting[
    language=python,linerange={56-93},
    caption=Класс интерфейса
    ]{1/inc/code.py}

\pagebreak
\section{Описание тестирования}

В таблице \ref{tabular:func_test} приведен функциональные тесты для алгоритмов
вычисления расстояния Левенштейна и Дамерау — Левенштейна.


\begin{table}[h]
    \centering
    \begin{tabular}{|c|c|c|}
        \hline
        \bfseries Строка 1  & \bfseries Строка 2 & \bfseries Ожидаемый результат
        \csvreader[no head]{1/inc/func_test.csv}{}
        {\\\hline \csvcoli&\csvcolii& \csvcoliii \ \csvcoliv}
        \\\hline
    \end{tabular}
    \caption{\label{tabular:func_test} Функциональные тесты}
\end{table}
