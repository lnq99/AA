file:///home/ql/5/AA/report/1/2-analysis.tex {"mtime":1601500268551,"ctime":1600807992274,"size":3610,"etag":"35obeeqic3ne","orphaned":false}
\chapter{Аналитический раздел}
\label{cha:analysis}

В данном разделе будет приведено описание алгоритмов.


\section{Описание алгоритмов}

\subsection{Расстояние Левенштейна}

Расстояние Левенштейна определяет минимальное количество операций,
необходимых для превращения одной последовательности символов в другую.
Разрешенные действия:

\begin{itemize}
    \setlength{\itemsep}{0em}
    \item вставка (I - insert)
    \item удаление (D - delete)
    \item замена (R - replace)
\end{itemize}


Расстояние Левенштейна между двумя строками $a, b$ задается выражением $lev_{a,b}(|a|,|b|)$ где
\\\\

$
lev_{a,b}(i,j) = \left\{
    \begin{array}{ll}
        max(i,j)        \hspace{5.5cm} if \ min(i,j) = 0\\
        min \left\{
            \begin{array}{ll}
                lev_{a,b}(i-1,j) + 1    \\
                lev_{a,b}(i,j-1) + 1    \hspace{2cm} otherwise\\
                lev_{a,b}(i-1,j-1) + 1_{(a_i \ne b_i)}\\
            \end{array}
            \right.\\
    \end{array}
\right.
$\\\\

где
$
1 _ {(a_i \ne b_i)} = \left\{
    \begin{array}{ll}
        0 \quad if \ a_i = b_i\\
        1 \quad otherwise\\
    \end{array}
    \right.
$\\

$lev_{a,b}(i,j)$ -
расстояние между первыми i символами строки a и первыми j символами строки b

\pagebreak
\subsection{Расстояние Дамерау-Левенштейна}

Если к списку разрешённых операций расстояния Левенштейна добавить
транспозицию (два соседних символа меняются местами),
получается расстояние Дамерау — Левенштейна.


\begin{itemize}
    \setlength{\itemsep}{0em}
    \item + транспозицию (T - transposition)
\end{itemize}

Расстояние  Дамерау-Левенштейна между двумя строками $a, b$ задается выражением $d_{a,b}(|a|,|b|)$ где
\\\\
$
d_{a,b}(i,j) = min\left\{
    \begin{array}{ll}
        0        \hspace{6cm} if \ i=j=0\\
        d_{a,b}(i-1,j) + 1    \hspace{2.8cm} if \ i>0\\
        d_{a,b}(i,j-1) + 1    \hspace{2.8cm} if \ j>0\\
        d_{a,b}(i-1,j-1) + 1_{(a_i \ne b_i)}
        \hspace{0.8cm} if \ i,j>0\\
        d_{a,b}(i-2,j-2) + 1
        \hspace{1.9cm} if \ i,j>1 \ and \\
        \hspace{6.2cm} a[i]=b[j-1] \ and \ a[i-1]=b[j]\\
    \end{array}
\right.
$\\\\

где
$
1 _ {(a_i \ne b_i)} = \left\{
    \begin{array}{ll}
        0 \quad if \ a_i = b_i\\
        1 \quad otherwise\\
    \end{array}
    \right.
$\\\\


Каждый рекурсивный вызов соответствует одному из случаев:

\begin{itemize}
    \setlength{\itemsep}{0em}
    \item $d_{a,b}(i-1,j) + 1$ соответствует удалению символа (из $a$ в $b$)
    \item $d_{a,b}(i,j-1) + 1$ соответствует вставке (из $a$ в $b$)
    \item $d_{a,b}(i-1,j-1) + 1_{(a_i \ne b_i)}$ соответствие или несоответствие, в зависимости от совпадения символов
    \item $d_{a,b}(i-2,j-2) + 1$ в случае перестановки двух последовательных символов
\end{itemize}

