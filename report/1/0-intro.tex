\chapter*{Введение}
\addcontentsline{toc}{chapter}{Введение}


Расстояние Левенштейна - минимальное количество односимвольных операций (вставки, удаления, замены), необходимых для превращения одной последовательности символов в другую.

Расстояние Левенштейна применяется в теории информации и
компьютерной лингвистике для:

\begin{itemize}
    \setlength{\itemsep}{0em}
    \item исправления ошибок в слове
    \item сравнения текстовых файлов утилитой diff и ей подобными
    \item в биоинформатике для сравнения генов, хромосом и белков
\end{itemize}


\textbf{Целью работы:} изучение метода динамического программирования на материале алгоритмов Левенштейна и
Дамерау-Левенштейна.

\textbf{Задачи работы:}

\begin{enumerate}
    \setlength{\itemsep}{0em}
    \item изучение алгоритмов Левенштейна и Дамерау-Левенштейна нахождения
расстояния между строками;
    \item применение метода динамического программирования для матричной
реализации указанных алгоритмов;
    \item получение практических навыков реализации указанных алгоритмов:
двух алгоритмов в матричной версии и одного из алгоритмов
в рекурсивной версии;
    \item сравнительный анализ линейной и рекурсивной реализаций выбранного
алгоритма определения расстояния между строками по затрачиваемым
ресурсам (времени и памяти);
    \item экспериментальное подтверждение различий во временной
    эффективности рекурсивной и нерекурсивной реализаций выбранного алгоритма
определения расстояния между строками при помощи разработанного
программного обеспечения на материале замеров процессорного
времени выполнения реализации на варьирующихся длинах строк;
    \item описание и обоснование полученных результатов в отчете о
выполненной лабораторной работе, выполненного как расчётно-пояснительная
записка к работе.
\end{enumerate}
